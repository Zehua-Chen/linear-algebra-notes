\section{Plane Equation}

  \begin{equation}
    a\left( x - x_{0} \right) + b\left( y - y_{0} \right) + c\left( z - z_{0} \right) + d = 0
  \end{equation}

  \begin{itemize}
    \item Two planes perpendicular to a third plane \textbf{are not always parallel}
    \item Two lines parallel to the same plane \textbf{are not always parallel}
    \item Two lines intersects, skew or parallel
    \item Two planes either intersects or parallel
  \end{itemize}

  \subsection{Parallel and Perpendicular}

    \begin{itemize}
      \item Two planes are parallel if their normal vectors are proportional
      \item Two planes are perpendicular if the dot products of their normal
      vectors is $ 0 $
    \end{itemize}

  \subsection{Parametric Equation}

    \begin{align*}
      x &= t + 2 \\
      y &= 2 - t \\
      z &= t
    \end{align*}

    \begin{itemize}
      \item Parametric equations express the coordinates of a point or vector
      \item To get the vector, derive each of the equations
      \begin{itemize}
        \item Ex. the vector of the equation above is $ \left< 1, -1, 1 \right> $
      \end{itemize}
    \end{itemize}


  \subsection{Linear Approximation}

    \begin{equation}
      l = f\left( a, b \right)
        + f_{x} \left( a, b \right) \Delta x
        + f_{y} \left( a, b \right) \Delta y
    \end{equation}

  \subsection{Tangent Plane}

    \begin{equation}
      z - f\left( a, b \right)
        = f_{x} \left( a, b \right) \Delta x
        + f_{y} \left( a, b \right) \Delta y
    \end{equation}