\section{Review}

  \subsection{Integration}

    \subsubsection{U-Subtitution}

      \begin{displaymath}
        \int \left( 3x^{2} + 2x \right) e^{\left( x^{3} + x^{2} \right)} dx
      \end{displaymath}

      \begin{align*}
        u &= x^{3} + x^{2} \\
        du &= 3x^{2} + 2x dx \\
        \frac{du}{3x^{2} + 2x} &= dx
      \end{align*}

      Plugin $ u $ and $ dx $ back in

      \begin{align*}
        &\int \left( 3x^{2} + 2x \right) e^{\left( x^{3} + x^{2} \right)} dx \\
        &= \int \left( 3x^{2} + 2x \right) e^{u} \frac{du}{3x^{2} + 2x} \\
        &= \int e^{u} du
      \end{align*}

      When dealing with definite integrals, be sure to replace $ a, b $ with
      $ u\left( a \right), u\left( b \right) $

    \subsubsection{Integration by Parts}

      \begin{equation}
        \int u dv = uv - \int v du
      \end{equation}

  \subsection{Estimation}

    \subsubsection{Riemann Sum}

      \begin{itemize}
        \item Consider the area under a curve as being made up of small rectangles.
        \item Take height of either the top left or top right corner and
        multiply the height by the width of the rectangles to get the areas
        \item Sum all the areas to get the area under the curve
      \end{itemize}

    \subsubsection{Midpoint Rule}

      \begin{itemize}
        \item Similar to rieman sum, but take the top middle of the rectangles
        to get the height, which is then used to calculate the area of the
        rectangles
      \end{itemize}