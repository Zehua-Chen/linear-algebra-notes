\section{Integration}

  \begin{align*}
    \int \int_{R} f\left( x, y \right) dA
      &= \int_{c}^{d} A\left( y \right) dy \\
      &= \int_{c}^{d} \left( \int_{a}^{b} f\left( x, y \right) dx \right) dy \\
  \end{align*}

  \begin{align}
    dA &= dx dy \\
    &= r dr d\theta
  \end{align}

  \begin{itemize}
    \item $ dA = r dr d\theta $ is used when in polar coordinates
  \end{itemize}

  \subsection{Reversing Order of Integration}

    \begin{equation}
      \int_{c}^{d} \left( \int_{a}^{b} f\left( x, y \right) dx \right) dy
        = \int_{a}^{b} \left( \int_{c}^{d} f\left( x, y \right) dy \right) dx
    \end{equation}

  \subsection{Contour Integrals}

    Given a vector function
    \begin{displaymath}
      F\left( x, y \right) = \left( P\left( x, y \right) + Q\left( x, y \right) \right)
    \end{displaymath}

    the contour integral of this vector function will be
    \begin{equation}
      \oint P dx + Q dy
    \end{equation}

    \subsubsection{Points/Lines and Contour Integrals}

      Given some points, and evaluate the contour integral

      \begin{displaymath}
        \oint ... dx + ... dy
      \end{displaymath}

      \begin{enumerate}
        \item Derive lines from the points (pay attention to direction,
        \textbf{conterclockwise} or \textbf{clockwise})
        \item If $ x $ does not change from one end of the lines to the other
        end, then plugin $ x $ as if they are numbers, and $ dx = 0 $
        \item If $ y $ does not change from one end of the lines to the other
        end, then plugin $ y $ as if they are numbers, and $ dy = 0 $
        \item Sum the results from all lines
      \end{enumerate}