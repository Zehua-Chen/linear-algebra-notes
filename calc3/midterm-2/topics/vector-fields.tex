\section{Vector Fields}

  For $ \mathbb{R}^{2} $, a vector field is a function
  $ f: \mathbb{R}^{2} \to \mathbb{R}^{2} $
  \begin{itemize}
    \item Input are coordinates on the grpah
    \item Output are the vectors are the coordinates
  \end{itemize}

  \subsection{Curve in Vector Field}

    If we represent $ C $ as
    \begin{align*}
      \vec{r}\left( t \right)
        &= \left( x\left(t\right), y\left(t\right) \right) \\
        &= x\left( t \right) \vec{i} + y\left( t \right) \vec{j} \\
      \vec{F}\left( x, y \right)
        &= P\left( x, y \right) \vec{l} + Q\left( x, y \right) \vec{j}
    \end{align*}

    $ P, Q $ describe the two components of a vector based on point.

  \subsection{Integration}

    Given
    \begin{itemize}
      \item $ C $: a curve in $ \mathbb{R}^{2} $
      \item $ \vec{F}: \mathbb{R}^{2} \to \mathbb{R}^{2} $: a vector field
      \item $ \vec{r}\left( a, b \right) \to \mathbb{R} $, a parameterization
      of $ C $
    \end{itemize}

    \begin{align}
      \int_{C} \vec{F} d\vec{r}
        &= \int_{a}^{b} \vec{F}\left( \vec{r} \left( t \right) \right)
        \cdot \vec{r}'\left( t \right) dt \\
        &= \int_{C} P dx + Q dy
    \end{align}

    \subsubsection{Integration and Tangent Vector}

      \begin{align*}
        \vec{T} &= \frac{\vec{r}'\left( t \right)}{\left| \vec{r}'\left( t \right) \right|} \\
        ds &= \vec{r}'\left( t \right)
      \end{align*}

      \begin{align*}
        \int_{C} \vec{F} \cdot d \vec{r}
        &= \int_{C} \vec{F} \cdot \vec{T} ds \\
        &= \int_{C} \vec{F} \cdot \vec{T} \frac{\left| \vec{r}'\left( t \right) \right| dt}{ds}
      \end{align*}

  \subsection{Conservative}

    If a vector field $ \vec{F} $ is conservative:

    \begin{align}
      \vec{F} &= \nabla f \\
      \int_{c} \vec{F} \cdot d \vec{r} &= f\left( B \right) - f\left( A \right)
    \end{align}

    where $ A, B $ are points in the vector field.

    \begin{itemize}
      \item When $ \vec{F} = \nabla f $ we call $ f $ a potential function
      for $ \vec{F} $ and we say $ \vec{F} $ is a conservative vector field
    \end{itemize}

    \paragraph{Find f from F}
    Given $ F = ai + bj $

    \begin{enumerate}
      \item Since $ \nabla f = F $, we know $ f_{x} = a $  and
      $ f_{y} = b $
      \item Integrate $ f_{x}, f_{y} $
    \end{enumerate}

    \subsubsection{Open and Connected}

      \begin{itemize}
        \item \textbf{Open}: an open shape is a shape such that you
        can find a radius and a point in the shape and draw a circle
        and the circle is in the shape
        \item \textbf{Connected}: for any two points in the shape, there is a
        curve that connects the two points
        \item \textbf{Simply Connected}: for any two points in the shape,
        there is a straight line that connects the two points
      \end{itemize}

      \begin{equation}
        \frac{dP}{dy} = \frac{dQ}{dx}
      \end{equation}

    \subsubsection{Path Independence}

      Given a vector field $ F $ and two different curves
      $ C_{1}, C_{2} $ that start and end at the same points $ A $ and $ B $,
      the vector field is independent if

      \begin{equation}
        \int_{C_{1}} F \cdot ds = \int_{C_{2}} F \cdot ds
      \end{equation}

    \subsubsection{Theorem A}

      A vector field $ \vec{F} $ on an \ul{open} and \ul{connected} region
      $ D $ in $ \mathbb{R}^{2} $ is conservative if and only if
      \begin{displaymath}
        \int_{C} \vec{F} \cdot d\vec{r}
      \end{displaymath}
      is path independent.

      \begin{itemize}
        \item Works with $ \mathbb{R}^{n} $
      \end{itemize}

    \subsubsection{Theorem B}

      $ \vec{F}\left( P, Q \right) $ on a simply connected open $ D $
      in $ \mathbb{R}^{2} $ is conservative if and only if
      \begin{equation}
        \frac{dP}{dy} = \frac{dQ}{dx}
      \end{equation}

      \begin{itemize}
        \item Have more complicated versions for $ \mathbb{R}^{n} $
      \end{itemize}
