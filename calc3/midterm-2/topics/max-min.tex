\section{Max and Min}

  A critical points can be any of the followings:
  \begin{itemize}
    \item Absolute/Local Min
    \item Absolute/Local Max
    \item Saddle
  \end{itemize}

  \subsection{Critical Points}

    A point $ \left( a, b \right) $ is a critical point if
    \begin{equation}
      \nabla f \left( a, b \right) = 0
    \end{equation}

    \subsubsection{Finding Critical Points}

      \begin{enumerate}
        \item In $ f_{x} $, try find values for $ y $; if not possible,
        find $ y $ in terms of $ x $
        \item In $ f_{y} $, plugin $ y $ in terms of $ x $ and find the values
        of x
        \item Plug the values of $ x $ to find the remaining values of $ y $
      \end{enumerate}

  \subsection{Local Extreme: Second Derivative Test}

    \begin{equation}
      D =
      \left|
        \begin{bmatrix}
          f_{xx} (a, b) & f_{xy} (a, b) \\
          f_{xy} (a, b) & f_{yy} (a, b)
        \end{bmatrix}
      \right|
    \end{equation}

    \begin{itemize}
      \item if $ D > 0 $
      \begin{itemize}
        \item $ f_{xx} (a, b) > 0 $, then $ (a, b) $ is a \textbf{local min}
        \item $ f_{xx} (a, b) < 0 $, then $ (a, b) $ is a \textbf{local max}
      \end{itemize}

      \item if $ D < 0 $, then $ (a, b) $ is a \textbf{saddle}
    \end{itemize}

  \subsection{Absolute Extreme: Extreme Value Theorem}

    Suppose $ f $ is continuous, on $ \left[ a, b \right] = \{ a \le x \le b \} $,
    then $ f $ has both an absolute minimum and an absolute maximum in
    $ \left[ a, b \right] $

    The absolute extremes occur at
    \begin{itemize}
      \item A critical point of $ f $
      \item One of $ a, b $
    \end{itemize}

    \subsubsection{Boundary Lines}

      For 3D graphs, the boundaries are not a point, but a line. Therefore,
      to find the absolute extremes, the following points on the line must be
      tested
      \begin{itemize}
        \item Start
        \item End
        \item Critical point of the line
      \end{itemize}

  \subsection{Constrained Optimization}

    \textbf{Constrained optimization}: finding the extreme values on $ f $,
    bound by (aka. subject to) $ g $

    \begin{itemize}
      \item The function to optimize:
      \begin{displaymath}
        f: \mathbb{R}^{n} \to \mathbb{R}
      \end{displaymath}

      \item The constraint: limits the possible inputs of $ f $
      \begin{displaymath}
        g: \mathbb{R}^{n - 1} \to \mathbb{R}
      \end{displaymath}
    \end{itemize}

    The maximum and minimum values happen at points where $ g $ and $ f $
    are \textbf{tangent}

    \subsubsection{Larange Multipliers}

      Given a function $ f $ and its constraint $ g $, and extreme values
      happen at points where $ f $ and $ g $ are tangent, the gradient values
      of $ g $ and $ f $ should at least point to the same direction.

      When two vectors to the same same direction, they are
      \textbf{proportional}.

      \begin{align}
        \nabla f\left( ... \right) &= \lambda \nabla g\left( ... \right) \\
        \nabla f\left( ... \right) - \lambda \nabla g\left( ... \right) &= 0
      \end{align}

      With this equation, we would be able to find the points where $ f $
      and $ g $ are tangent (touch) each other and these points
      \textbf{can be extreme values}

    \subsubsection{Solve Lagrange Multipliers}

      \begin{displaymath}
        ax = \lambda bx, cy = \lambda dy, ...
      \end{displaymath}

      Try find $ \lambda $, and then use $ \lambda $ to find the values
      of $ x, y, ... $, which are the cirtical points.