\section{Max and Min}

  A critical points can be any of the followings:
  \begin{itemize}
    \item Absolute/Local Min
    \item Absolute/Local Max
    \item Saddle
  \end{itemize}

  \subsection{Critical Points}

    A point $ \left( a, b \right) $ is a critical point if
    \begin{equation}
      \nabla f \left( a, b \right) = 0
    \end{equation}

    \subsubsection{Finding Critical Points}

      \begin{enumerate}
        \item In $ f_{x} $, try find values for $ y $; if not possible,
        find $ y $ in terms of $ x $
        \item In $ f_{y} $, plugin $ y $ in terms of $ x $ and find the values
        of x
        \item Plug the values of $ x $ to find the remaining values of $ y $
      \end{enumerate}

  \subsection{Local Extreme: Second Derivative Test}

    \begin{equation}
      D =
      \left|
        \begin{bmatrix}
          f_{xx} (a, b) & f_{xy} (a, b) \\
          f_{yx} (a, b) & f_{yy} (a, b)
        \end{bmatrix}
      \right|
    \end{equation}

    \begin{itemize}
      \item if $ D > 0 $
      \begin{itemize}
        \item $ f_{xx} (a, b) > 0 $, then $ (a, b) $ is a \textbf{local min}
        \item $ f_{xx} (a, b) < 0 $, then $ (a, b) $ is a \textbf{local max}
      \end{itemize}

      \item if $ D < 0 $, then $ (a, b) $ is a \textbf{saddle}
      \item \ul{This method only works if the point is a \textbf{critical point}}
    \end{itemize}

  \subsection{Absolute Extreme: Extreme Value Theorem}

    Given a \textbf{continuous} $ f $, and a \textbf{bounded and closed}
    shape $ D $ (domain in $ \mathbb{R} $), $ f $ has both an absolute minimum
    and an absolute maximum in $ D $ at the followings:
    \begin{itemize}
      \item A critical point of $ f $
      \item On the boundaries of $ D $
    \end{itemize}

    \subsubsection{Bound and Closed}

      \begin{itemize}
        \item \textbf{Bound}: $ D $'s values are in a non-infinite range
        \item \textbf{Closed}: if the shape has boundary points are included
        \begin{itemize}
          \item \textbf{Not Closed}: $ x^{2} + y^{2} < 9 $
          \item \textbf{Closed}: $ x^{2} + y^{2} \le 9 $
          \item \textbf{Closed}: $ x^{2} + y^{2} \ge 9 $
        \end{itemize}

        \item \textbf{Bound} and \textbf{closed} are independent
      \end{itemize}

      \subsubsection{Boundary Points}

        A point on a function such that a circle with an arbitrary radius
        can be drawn at the point and still includes area on both sides of the
        function.

        In other words, boundary points are points on the line formed by the
        function.

      \subsubsection{Finding the Absolute Extremes}

        \begin{enumerate}
          \item Find all critical points of $ f $ in $ D $ and determine
          the values at these critical points
          \item Find the extremes of the function on the boundary:
          \begin{itemize}
            \item Start and end of the boundary, if applicable
            \item Critial points of the boundary
          \end{itemize}

          \item The max and min of the previous steps are the extremes
        \end{enumerate}

  \subsection{Constrained Optimization}

    \textbf{Constrained optimization}: finding the extreme values on $ f $,
    bound by (aka. subject to) $ g $

    \begin{itemize}
      \item The function to optimize:
      \begin{displaymath}
        f: \mathbb{R}^{n} \to \mathbb{R}
      \end{displaymath}

      \item The constraint: limits the possible inputs of $ f $
      \begin{displaymath}
        g: \mathbb{R}^{n - 1} \to \mathbb{R}
      \end{displaymath}
      \begin{itemize}
        \item Can also be $ D $ from \textbf{extreme value theorem}
      \end{itemize}
    \end{itemize}

    The maximum and minimum values happen at points where $ g $ and $ f $
    are \textbf{tangent}

    \subsubsection{Larange Multipliers}

      Given a function $ f $ and its constraint $ g $, and extreme values
      happen at points where $ f $ and $ g $ are tangent, the gradient values
      of $ g $ and $ f $ should at least point to the same direction.

      When two vectors to the same same direction, they are
      \textbf{proportional}.

      \begin{align}
        \nabla f\left( ... \right) &= \lambda \nabla g\left( ... \right) \\
        \nabla f\left( ... \right) - \lambda \nabla g\left( ... \right) &= 0
      \end{align}

      With this equation, we would be able to find the points where $ f $
      and $ g $ are tangent (touch) each other and these points
      \textbf{can be extreme values}

      \begin{itemize}
        \item Still needs to compute $ \nabla f $ to find all extreme values
      \end{itemize}

    \subsubsection{Solve Lagrange Multipliers}

      \begin{displaymath}
        ax = \lambda bx, cy = \lambda dy, ...
      \end{displaymath}

      Try find $ \lambda $, and then use $ \lambda $ to find the values
      of $ x, y, ... $, which are the cirtical points.

      \paragraph{Ex}
      \begin{displaymath}
        \begin{cases}
          2x + 2 = \lambda 2 x \\
          2y + 2 = \lambda 2 y
        \end{cases}
      \end{displaymath}

      \begin{align*}
        \lambda &= \frac{x + 1}{x} = 1 + \frac{1}{x} \\
        \lambda &= \frac{y + 1}{y} = 1 + \frac{1}{y} \\
        x &= y
      \end{align*}

      Given
      \begin{displaymath}
        x^{2} + y^{2} = 9
      \end{displaymath}

      \begin{align*}
        2x^{2} &= 9 \\
        x^{2} &= \frac{9}{2} \\
        x &= \pm \frac{3\sqrt{2}}{2}
      \end{align*}

  \subsection{Extreme Value and Tangent Plane / Linear Approximation}

    \begin{itemize}
      \item At a \textbf{local max}, the tangent plant is horizontal,
      and at the points around the local max, the tangent plane lies
      above the original equation;
      \ul{therefore, at local max, linear approximation is bigger than
      the original equation};
      \item At a \textbf{local min}, the tangent plant is horizontal,
      and at the points around the local max, the tangent plane lies
      below the original equation;
      \ul{therefore, at local min, linear approximation is smaller than
      the original equation};
    \end{itemize}