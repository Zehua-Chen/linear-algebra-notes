\section{Max and Min}

  A critical points can be any of the followings:
  \begin{itemize}
    \item Absolute/Local Min
    \item Absolute/Local Max
    \item Saddle
  \end{itemize}

  \subsection{Critical Points}

    A point $ \left( a, b \right) $ is a critical point if
    \begin{equation}
      \nabla f \left( a, b \right) = 0
    \end{equation}

  \subsection{Local Extreme: Second Derivative Test}

    \begin{equation}
      D =
      \left|
        \begin{bmatrix}
          f_{xx} (a, b) & f_{xy} (a, b) \\
          f_{xy} (a, b) & f_{yy} (a, b)
        \end{bmatrix}
      \right|
    \end{equation}

    \begin{itemize}
      \item if $ D > 0 $
      \begin{itemize}
        \item $ f_{xx} (a, b) > 0 $, then $ (a, b) $ is a \textbf{local min}
        \item $ f_{xx} (a, b) < 0 $, then $ (a, b) $ is a \textbf{local max}
      \end{itemize}

      \item if $ D < 0 $, then $ (a, b) $ is a \textbf{saddle}
    \end{itemize}

  \subsection{Absolute Extreme: Extreme Value Theorem}

    Suppose $ f $ is continuous, on $ \left[ a, b \right] = \{ a \le x \le b \} $,
    then $ f $ has both an absolute minimum and an absolute maximum in
    $ \left[ a, b \right] $

    The absolute extremes occur at
    \begin{itemize}
      \item A critical point of $ f $
      \item One of $ a, b $
    \end{itemize}