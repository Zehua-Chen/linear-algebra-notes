\documentclass[11pt]{article}
\usepackage{newtxtext}
\usepackage{newtxmath}
\usepackage{amsmath}

\DeclareMathOperator{\proj}{proj}
\DeclareMathOperator{\comp}{comp}

\title{MATH 241 Midterm 1 Review}

\setlength{\parindent}{0pt}

\begin{document}
  \maketitle

  \paragraph{Dot Product}
  \begin{align}
    \vec{v} \cdot \vec{w} &= \left| \vec{v} \right| \left| \vec{w} \right| \cos\left( \theta \right)
  \end{align}

  \paragraph{Projection}
  \begin{align}
    \comp_{\vec{v}} \vec{w} &= \frac{ \vec{v} \cdot \vec{w} }{ \left| \vec{v} \right| } \\
    \proj_{\vec{v}} \vec{w} &= \frac{ \vec{v} \cdot \vec{w} }{ \left| \vec{v} \right|^{2} } \vec{v}
  \end{align}

  \paragraph{Corss Product}
  \begin{align}
    \vec{v} \times \vec{w} &= \left| \vec{v} \right| \left| \vec{w} \right| \sin\left( \theta \right)
  \end{align}

  \begin{itemize}
    \item Two unit vectors are perpendicular if their cross product is 1
    \item Two unit vectors are perpendicular if their cross product is 0
  \end{itemize}

  \paragraph{Plane Equation}
  \begin{equation}
    ax + by + cz + d = 0
  \end{equation}

  \begin{itemize}
    \item Two planes perpendicular to a third plane \textbf{are not always parallel}
    \item Two lines parallel to the same plane \textbf{are not always parallel}
    \item Two lines intersects, skew or parallel
    \item Two planes either intersects or parallel
  \end{itemize}

    \subparagraph{Parametric Equation}

    \begin{align*}
      x &= t + 2 \\
      y &= 2 - t \\
      z &= t
    \end{align*}

    \begin{itemize}
      \item Parametric equations express the coordinates of a point or vector
      \item To get the vector, derive each of the equations
      \begin{itemize}
        \item Ex. the vector of the equation above is $ \left< 1, -1, 1 \right> $
      \end{itemize}
    \end{itemize}

  \paragraph{Circle Equation}
  \begin{equation}
    \left( x - x_{0} \right)^{2}
      + \left( y - y_{0} \right)^{2}
      + \left( z - z_{0} \right)^{2}
      = r^{2}
  \end{equation}

  \paragraph{Linear Approximation}
  \begin{equation}
    l = f\left( a, b \right)
      + f_{x} \left( a, b \right) \Delta x
      + f_{y} \left( a, b \right) \Delta y
  \end{equation}

  \paragraph{Tangent Plane}
  \begin{equation}
    z - f\left( a, b \right)
      = f_{x} \left( a, b \right) \Delta x
      + f_{y} \left( a, b \right) \Delta y
  \end{equation}

  \paragraph{Chain Rule}
  \begin{align}
    h(t) &= f(x(t), y(t)) \\
    h'(0) &= f_{x}(x(0), y(0)) x'(a) + f_{y}(x(0), y(0)) y'(a)
  \end{align}

  \paragraph{Level Set Problems}
  \begin{itemize}
    \item $ f_{a} $ what happens to $ f_{a} $
    \item $ f_{ab} $ what happens to $ f_{b} $ when $ a $ changes
  \end{itemize}

  \paragraph{Graph Problems}
  \begin{enumerate}
    \item Signs
    \item X, Y planes
    \item Slices of cos, sin, tan graphs are periodic
  \end{enumerate}

  \paragraph{Colaence Theorem}
  \begin{equation}
    f_{xy} = f_{yx}
  \end{equation}

\end{document}