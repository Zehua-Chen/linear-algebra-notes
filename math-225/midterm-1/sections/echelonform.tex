\section{Echelon Form}

\begin{itemize}
  \defitem{Pivot}
  {The leading nonzero element in a row};
  \begin{itemize}
    \item The pivot positions in a matrix are determined completely by the 
    positions of the leading entries in the nonzero rows of any echelon form 
    obtained from the matrix
  \end{itemize}
  
  \defitem{Echelon Form}{}
  \begin{itemize}
    \item All nonzero rows are above zero rows;
    \item The leading coefficient of a nonzero row is always strictly to the 
    right of the leading coefficient of the row above it;
    \item The echelon form of a matrix is not unique;
  \end{itemize}
  
  \defitem{Reduced Echelon Form}{}
  \begin{itemize}
    \item An echelon form;
    \item All the pivots are zero and the only nonzero element in its column;
    \item The reduced echelon form of a matrix is unique;
  \end{itemize}
  
  \defitem{Pivot Column}{the column where the pivot is};
  \defitem{Forward Phase}{the process to reduce a matrix to echelon form};
  \defitem{Backward Phase}{the process to reduce a matrix to reduced echelon form};
  
  \defitem{General Solution}
  {a general solution of a system is an explicit description of all solutions of the system};
  
  \defitem{Undetermined System}
  {a system of linear equations where there are fewer equations than variables}
  \begin{itemize}
    \item Cannot have a unique solution; If the solution is consistent, 
    then there will be infinitely many solutions due to free variables; 
    if the solution is inconsistent, there will not be solutions;
  \end{itemize}
\end{itemize}