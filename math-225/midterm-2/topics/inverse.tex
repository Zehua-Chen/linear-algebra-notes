\section{Inverse}

\begin{definition}
  Say a $ n \times n $ matrix $ A $ is inversible if 
  \begin{displaymath}
    \exists B, A \cdot B = I = B \cdot A
  \end{displaymath}
\end{definition}

\subsection{Finding Inverse}

  \paragraph{$ n = 1 $}
  \begin{equation}
    \begin{bmatrix}
      a
    \end{bmatrix}^{-1}
    =
    \begin{bmatrix}
      a^{-1}
    \end{bmatrix}
  \end{equation}
  
  \paragraph{$ n = 2 $}
  \begin{equation}
    \begin{bmatrix}
      a & b \\ 
      c & d
    \end{bmatrix}^{-1}
    =
    \begin{bmatrix}
      d & -b \\ 
      -c & a
    \end{bmatrix}
    \times \frac{1}{a \times d - b \times c}
  \end{equation}
  
  \paragraph{General}
  \begin{equation}
    \begin{bmatrix}
      A I
    \end{bmatrix}
    =
    \begin{bmatrix}
      I A^{-1}
    \end{bmatrix}
  \end{equation}
  
  \subsubsection{Questions from Assignments}
  \begin{enumerate}
    \item Question 10 of Assignment 6;
  \end{enumerate}

\subsection{Properties}
  
  \begin{enumerate}
    \item \highlight{If $ a \times d - b \times c \ne 0 $ then 
    the matrix is invertible};
    
    \item The \highlight{inverse of inverse is the original matrix}
    \begin{displaymath}
      A = \left( A^{-1} \right)^{-1}
    \end{displaymath}
    
    \item $ A \cdot x = b $ only has one solution;
    \begin{itemize}
      \item Therefore, $ A \cdot x = 0 $ only has a \textbf{trivial solution};
    \end{itemize}
    
    \item If $ \left| A \right| = 0 $, then $ A $ is not invertible;
    \item If $ A $ is linearly depedent, then $ A $ not invertible;
    
    \item If $ A $ is inversible, then \highlight{$ A^{-1} $ is inversible};

    \item Given $ A, $ that are both inversible:
    \begin{align*}
      I &= A \cdot B \cdot B^{-1} \cdot A^{-1} \\ 
      &= \left( A \cdot B \right) \cdot \left( B^{-1} \cdot A^{-1} \right)
    \end{align*}
    
    \item Transpose and inverses 
    \begin{align*}
      I^{T} &= \left( A \cdot A^{-1} \right) \\ 
      &= \left( A^{-1} A \right)^{T}
    \end{align*}
    \begin{align*}
      I &= A^{T} \cdot \left( A^{-1} \right)^{T} \\ 
      &= \left( A^{-1} \right)^{T} \cdot A^{T}
    \end{align*}
    \begin{align*}
      \left( A^{T} \right)^{-1} &= \left( A^{-1} \right)^{T} \\ 
      \left( \left( A \cdot B \right)^{T} \right)^{-1} &= 
      \left( A^{T} \right)^{-1} \cdot \left( B^{T} \right)^{-1}
    \end{align*}
    
    \item Given $ A $ is inversible:
    \begin{enumerate}
      \item $ A \cdot x = b $ is \highlight{consistent 
      $ \forall b \in \mathbb{R}^{n} $}
      
      \item $ A \cdot x = 0 $ has \highlight{only \textbf{trivial solution}};
      
      \item \highlight{Columns and rows} of $ A $ are 
      linearly independent vectors in $ \mathbb{R}^{n} $
      
      \item $ A $ is linearly independent;
      
      \item $ \left[ A b \right] $ \highlight{never reduces to 
      $ \left[ 0...0 x \right] $ (augmented matrix)};
      
      \item After being reduced to echelon forms, 
      \highlight{all rows and columns of $ A $ have pivots};
      
      \item The \highlight{reduced echelon form of $ A $ is $ I $}
    \end{enumerate}
  \end{enumerate}