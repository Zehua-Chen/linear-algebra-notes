\section{Matrix Equation}
  
  \begin{itemize}
    \item To show that the equation $ A \vec{x} = b $ does not have solution 
    for some $ b $, \textbf{reduce $ A $ to show $ A $ does not have a pivot
    position in every row};
    \begin{itemize}
      \item In $ A \vec{x} = b $, if the reduced echelon form of $ A $ has a pivot in 
      every row, then the system is consistent;
      \begin{itemize}
        \item If $ A $ has a pivot in every row, then the augmented matrix cannot
        have rows like $ \left[ 0 0 0 1 \right] $
      \end{itemize}
    \end{itemize}
    
    \item Any linear combination of vectors can be always written in the form of 
    $ A \vec{x} $;
    \begin{itemize}
      \item The matrix $ A $ is the coefficients of the systems of vectors;
    \end{itemize}
    
    \item Let $ A $ be $ m \times n $ matrix, then the following statements 
    are either all true or all false
    \begin{enumerate}
      \item For each $ b \in R^{m} $, the equation $ A \vec{x} = b $ has a solution;
      \item $ \forall b \in R^{m} $, is a linear combination of the columns of $ A $;
      \item The column of $ A $ spans $ R^{m} $;
      \item $ A $ has a pivot position in every row;
    \end{enumerate}
    
    \item A solution has a \textbf{non-trivial solution} if there is \textbf{at least one
    free variable in its solution};
  \end{itemize}

  \subsection{Matrix Equation}
  
    \begin{equation}
      \begin{bmatrix}
        a_{1} & a_{2}  \\ 
        a_{3} & a_{4}  \\ 
        a_{5} & a_{6} 
      \end{bmatrix}
      \times
      \begin{bmatrix}
        x_{1} \\ 
        x_{2} 
      \end{bmatrix}
      = 
      \begin{bmatrix}
        b_{1} \\ 
        b_{2} \\ 
        b_{3}
      \end{bmatrix}
    \end{equation}
    
    \begin{itemize}
      \item Number of columns in $ A $ must equal number of rows in $ \vec{x} $
      for the solution to be defined;
    \end{itemize}
    
  \subsection{Vector Equation}
    
    \begin{equation}
      x_{1} \times 
      \begin{bmatrix}
        a_{1} \\ 
        a_{3} \\ 
        a_{5}
      \end{bmatrix}
      + 
      x_{2} \times 
      \begin{bmatrix}
        a_{2} \\ 
        a_{4} \\ 
        a_{6}
      \end{bmatrix}
      = 
      \begin{bmatrix}
        b_{1} \\ 
        b_{2} \\ 
        b_{3}
      \end{bmatrix}
    \end{equation}
  
    \begin{itemize}
      \item Another way of writing matrix equation;
    \end{itemize}
  
  \subsection{Row Vector Rule}
  
    \begin{itemize}
      \item For $ A \vec{x} $ to be defined, the number of colunms in $ A $ has to match the amount of 
      entries in $ \vec{x} $;
    \end{itemize}
  
  \subsection{Homogenous Equation}
  
    \begin{equation}
      A \vec{x} = 0
    \end{equation}
    
    \begin{itemize}
      \item \textbf{Homogenous equations} are \textbf{always consistent}, since 
      in $ A \vec{x} = 0 $, there will always be at least one solution 
      $ \vec{x} = 0 $;
      
      \item Homogenous equations gives \textbf{implicit solution set} $ \vec{x} = 0 $;
      \textbf{Solving the equation} gives the \textbf{explicit solution set} 
      $ \vec{x} = 0 $;
      
      \item Homogenous equation always has at least one \textbf{trivial solution},
      $ \vec{x} = 0 $;
    \end{itemize}
