\section{Vectors}
  
\subsection{Linear Combinations}

\paragraph{Graphical Thinking}
  
  \begin{itemize}
    \item A vector is a point;
    \item A vector multiplied by a weight is a point \textbf{stretched};
    \item A few cectors multiplied by a few weights added together can reach a point,
    which is called \textbf{linear combination};
    \item \textbf{Span} is the collection of all possible \textbf{linear combinations},
    in another word, all possible points reachable by adding a set of vectors,
    multiplied by the correct weight;
  \end{itemize}

  \paragraph{Mathmetic Thinking}

  \begin{itemize}
    \defitem{Linear Combination}
    {An expression constructed from a set of terms by multiplying each term
    by a constant and adding the results};
    \begin{itemize}
      \item To say $ b $ is a linear combination of $ a_{1}, a_{2}, a_{1} $
      is to say that there are weights (\textbf{multipliers}) $ x_{1}, x_{2}, x_{3} $ such that 
      $ a_{1} \cdot x_{1} + a_{2} \cdot x_{2} + a_{3} \cdot x_{3} = b $;
      \begin{displaymath}
        a_{1} =
        \begin{bmatrix}
          ? \\
          ? \\ 
          ? \\
        \end{bmatrix},
        a_{2} = 
        \begin{bmatrix}
          ? \\
          ? \\ 
          ? \\
        \end{bmatrix},
        a_{3} = 
        \begin{bmatrix}
          ? \\
          ? \\ 
          ? \\
        \end{bmatrix}
      \end{displaymath}
    \end{itemize}
    
    \defitem{Span}{All possible linear combinations};
    \item A system has solution if $ b \in span \{ a_{1} ... a_{n} \} $
    \begin{itemize}
      \item In another word, $ b $ is one of the possible linear combinations,
      which means that there are values of $ x_{1} ... x_{n} $ such that:
      \begin{displaymath}
        x_{1} \cdot a_{1} + ... + x_{n} \cdot a_{n} = b
      \end{displaymath}
      which means that there is a solution;
    \end{itemize}
  \end{itemize}
  

  
\subsection{Span in Different Dimensions}

  \subsubsection{Two Dimension}
  
  \begin{align}
    span \{ \vec{a} \} &= \mathbb{R} \\ 
    span \{ \vec{a_{1}}, \vec{a_{2}} \} &= 
    \begin{cases}
      \mathbb{R} \\ 
      \mathbb{R}^{2}
    \end{cases}
    \\
    span \{ \vec{a_{1}}, \vec{a_{2}}, ... \} &= 
    \begin{cases}
      \mathbb{R} \\ 
      \mathbb{R}^{2}
    \end{cases}
  \end{align}
  \begin{itemize}
    \item By adding two vectors (\textbf{esentially two points}) which are multiplied by the 
    correct weight $ \{ x_{1}, x_{2}, ... \} $, any points
    on the plane can be reached, denoted by $ \mathbb{R}^{2} $;
  \end{itemize}
  \subsubsection{Three Dimension}
  
  \begin{align}
    span \{ \vec{a} \} &= \mathbb{R} \\ 
    span \{ \vec{a_{1}}, \vec{a_{2}} \} &= 
    \begin{cases}
      \mathbb{R} \\ 
      \mathbb{R}^{2}
    \end{cases}
    \\
    span \{ \vec{a_{1}}, \vec{a_{2}}, \vec{a_{3}} \} &= 
    \begin{cases}
      \mathbb{R} \\ 
      \mathbb{R}^{2} \\ 
      \mathbb{R}^{3}
    \end{cases}
    \\
    span \{ \vec{a_{1}}, \vec{a_{2}}, \vec{a_{3}}, ... \} &= 
    \begin{cases}
      \mathbb{R} \\ 
      \mathbb{R}^{2} \\ 
      \mathbb{R}^{3}
    \end{cases}
  \end{align}